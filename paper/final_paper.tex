\documentclass[11pt,twocolumn]{article}

\usepackage[margin=1in]{geometry}
\usepackage{booktabs}
\usepackage{cite}
\usepackage{fancyref}
\usepackage{tablefootnote}
\usepackage{url}

\begin{document}
\title{FBFS:\@ The Facebook Filesystem}
\author{Andy Russell}
\date{May 14, 2014}
\maketitle

\begin{abstract}

Facebook allows users to search an incredible amount of information, including
status updates, photos and ``likes''. However, this information is only
accessible by clicking through multiple Web pages, and searching for a specific
piece of information is sometimes difficult through Facebook's Web search.

In this paper, I describe FBFS, the Facebook Filesystem, a filesystem that
provides access to a Facebook user's information. The implementation presents
Facebook information such as the status updates and photos of friends as actual
files on the user's filesystem. An added benefit of the filesystem is that many
UNIX tools such as \texttt{grep} and \texttt{find} are available to search for
Facebook content using regular expressions or other advanced queries, something
not possible with Facebook's own Web search. FBFS leverages Facebook's API (the
Graph API) to provide information when needed, without storing Facebook data on
a user's computer. I have taken care to provide a useful interface that takes
advantage of many filesystem features to provide a positive user experience.

\end{abstract}

\section{Introduction}

Facebook is a social networking website with more than 1 billion active monthly
users~\cite{aboutfb}. Facebook's popularity brings a diverse set of users with
it. These unique users interact with the website in many different ways:
through the Web interface on their computers, via apps on their tablets or
smartphones, or with text messages on traditional cell phones. For those who
use Facebook on their computer, there are also various ways of searching the
vast amount of information available: a user might click through links in the
web interface until their query is resolved, or utilize the search bar at the
top of the page.

The goal of FBFS is to provide an alternative interface to Facebook by
simulating a filesystem on a user's computer. This allows the user to access
the information that is available to him or her on Facebook while taking
advantage of the familiar hierarchical interface of a filesystem.

\section{Related Work}

FBFS is not the first project of its kind. I found two similar projects in my
initial research.

The first project, ``Facebook Filesystem'', is mainly
tongue-in-cheek\cite{fbfsperl}. The filesystem is implemented in Perl. The
program allows users to access their Facebook photos through a filesystem
interface. Additionally, the system allows users to add comments to the photos,
which are then embedded in the photo's binary data. However, this creates a
problem when the photo is uploaded back to Facebook. Because Facebook
compresses photos on upload, the binary data appended to the end of the photo
corrupts the rest of the image when displayed on the site. Though the
filesystem is intended as a joke, it raises an interesting question about the
design of a Facebook Filesystem: the question of how to deal with metadata. I
did not install this filesystem to my computer, as the author himself warns
that the filesystem is likely to corrupt your Facebook photos.

The second project I examined was also called ``FBFS''\cite{fbfsc}. This
filesystem is implemented in C. It has similar goals to my implementation of
FBFS:\@ the earlier program aims to provide an interface that allows UNIX
utilities to interact with information on Facebook in a meaningful way.
However, the implementation differs from mine in a number of ways. It requires
a user to run a Ruby Web server to create an access token, which the user then
must store in a predetermined directory. I disagree with this method of storing
the access token (explained further in Section~\ref{sec:design}). In addition,
the actual file system layout of the C implementation mirrors the Facebook API
more closely, which can make it more difficult for the user to interact with
the filesystem, because they have to remember the structure of the Graph API in
order to interact with the system. Though I attempted to install FBFS to my
computer, I could not get the binary to link properly.

Both implementations provided useful insight into some of the design decisions
that I had to make for FBFS\@.


\section{Background}
\label{sec:background}

\subsection{The Graph API}
\label{subsec:thegraphapi}

Facebook provides a powerful API for third-party applications to interact with
their website, hereafter referred to as the Graph API\cite{graphapi}. The Graph
API is structured in terms of ``nodes'' and ``edges'' though they do not mean
quite the same thing as the graph terminology they are inspired by. The API is
HTTP-based and responds with JSON, allowing developers to use existing tools to
interact with the API in a language-independent way. The syntax for requests
is summarized in Figure~\ref{fig:request}.

\begin{figure*}
\caption{Graph API Request Syntax}
\begin{center}
    \texttt{GET https://graph.facebook.com/<\textit{node}>/<\textit{edge}> HTTP/1.1}
\end{center}
\label{fig:request}
\end{figure*}

A node represents an object on Facebook, and an edge represents some kind of
information about that node. Additional parameters for the request may be added
on as key-value pairs at the end of the URL\@. If the request requires special
permissions that were requested when the application received the access token
(discussed in Section~\ref{subsec:facebookapplications}), the access token must
be appended at the end of the request as a URL parameter.

Upon receiving and processing a request, the API will respond with a JSON
object that contains a single field ``data'', that maps to object representing
the information requested. If the request is malformed or the application that
spawned the request has insufficient permissions to access the data requested,
the response will instead have a single field ``error'', that maps to an
object containing an error code and a brief message. I have included a few
examples of API requests and responses in Table~\ref{tab:request-response} to
help illustrate typical API interaction. Request headers, the hostname, the
HTTP version, and the access token parameter are all omitted from the request
for simplicity, though each is required when actually interacting with the
API\@.

\begin{table*}[h]
\caption{Sample Graph API Requests}
\centering
\begin{tabular}{ll}\toprule
    Request & Response \\\midrule
    \texttt{GET /me/statuses} &
\begin{minipage}{3in}
\begin{verbatim}
{"data": [
    {
        "message": "Hello World!",
        "id": "123456789",
        ... other fields ...
    },
    ... other status objects ...
    ]
}
\end{verbatim}
\end{minipage} \\\midrule
\texttt{GET /123456789} &
\begin{minipage}{3in}
\begin{verbatim}
{"data": {
    "message": "Hello World!",
    "id": "123456789",
    ... other fields ...
    }
}
\end{verbatim}
\end{minipage} \\\bottomrule
\end{tabular}
\label{tab:request-response}
\end{table*}

I have summarized some further examples of API requests and the effect they
have on Facebook in Table~\ref{tab:requests}. Though the API is quite
expressive, it is possible to obtain further granularity in the request by
using FQL, a SQL-like query language that allows complicated queries against
various ``tables'' such as friends and photos.

\begin{table*}[h]
\centering
\caption{Graph API Examples}
\begin{tabular}{lp{7cm}} \toprule
Request & Action \\\midrule
\texttt{GET /me} &
Returns the current user's basic profile information. \\

\texttt{GET /me/friends} &
Returns a list of objects containing IDs, names, etc.\ of the current user's
friends. \\

\texttt{GET /\textit{friend-id}/statuses?fields=message} &
Returns an array of objects containing only the text of the last 25 statuses
made by a friend with ID \textit{friend-id}. \\

\texttt{POST /me/feed?message=\textit{status}} &
Posts a status on the current user's feed with the text \textit{status}. \\

\texttt{DELETE /\textit{photo-id}} &
Deletes a photo with the id \textit{photo-id} from the current user's profile.
\\\bottomrule
\end{tabular}
\label{tab:requests}
\end{table*}

\subsection{Facebook Applications}
\label{subsec:facebookapplications}

For a program to communicate with Facebook, it must be associated with a
Facebook application. A programmer creates an application by logging into
Facebook's developer portal and choosing a name. From here, the developer can
access the ``app ID'', which is required to identify a client at login, and the
``app secret'', which is used for requesting special application-level
permissions from a client.

Once the Facebook application has been created, the client login process is
quite simple. First, the client directs a browser to a URL containing the
client ID and any additional permissions that the application needs to
function. The user is prompted to enter their username and password. If the
login is successful, then the user reviews the permissions requested by the
program and either accepts or declines the request to install the application.
The browser is automatically redirected to a URL that indicates whether the
login was successful or not. If successful, the URL will contain an access
token that is required for every login.

Facebook deals with permissions in a very flexible way. An application may
request additional permissions during normal execution by following the same
protocol as the login process. A user may choose to accept or deny these
additional permissions, and Facebook suggests that applications only request
permissions that are absolutely necessary. In addition, there are permissions
that a user may choose to accept or deny for all applications, such as allowing
access to their statuses from friends' applications.

\section{Design}
\label{sec:design}

FBFS is designed to be as user-friendly as possible. This is achieved in part
by adhering to the UNIX philosophy. To satisfy both constraints, there are a
number of design trade-offs that were made with the goal of making the
filesystem easy to use.

One of the problems encountered by this filesystem is the incredible amount of
metadata that each object on Facebook has associated with it. For example, a
photo album contains photos, each of which has comments, which have likes, etc.
In the initial phases of design, I considered storing this metadata in a
structured file containing human-readable markup such as YAML~\cite{yaml}.
However, such a structure would probably make it harder to search the files or
use them as input to programs. This also would violate the tenet of the UNIX
philosophy which states that data should be stored in flat text
files~\cite{Gancarz:1995:UP:193178}. Therefore, FBFS compromises on this point
and chooses to store Facebook information such as the text of a status inside a
flat file, while keeping comments in a separate, structured file.

Another design tradeoff that was necessary was a deviation from the Graph API
structure. Though the Graph API provides a hierarchical structure similar to a
filesystem in many cases, there are a number of ways in which it differs. For
example, consider that any object on Facebook may be used as a node. It would
be both unwieldy for a user and impractical for the developer to fill the root
directory with every node that could be available. Therefore, I have decided to
use the various edges of the graph (statuses, photos, friends, etc.) as folders
stored in the root directory. This makes it easy for a user to see all of their
statuses by listing the status directory. The mapping of various API requests
to filesystem operations is summarized in Table~\ref{tab:fsops}.

\begin{table*}[h]
\centering
\caption{REST vs.\ Directory Hierarchy}
\begin{tabular}{ll} \toprule
Graph API & FBFS \\\midrule
\texttt{GET /\textit{user-id}/statuses} &
\texttt{ls -l /friends/Andy\char092~Russell/status} \\
\texttt{GET /\textit{status-id}} &
\texttt{cat /friends/Andy\char092~Russell/status/\textit{status-id}} \\
\texttt{POST /me/feed?message=\textit{status}} & \texttt{echo "\textit{status}" > /status/post} \\ % chktex 18
\texttt{DELETE /\textit{photo-id}} & \texttt{rm /photos/\textit{photo-id}} \\\bottomrule
\end{tabular}
\label{tab:fsops}
\end{table*}

FBFS requires a user to login every time they mount the filesystem. This
ensures that the access token is fresh and valid, but it also burdens the user
with entering their username and password more often than is convenient. Though
it would be possible to store the access token between sessions and
automatically log the user in, this would create a security risk. A program
could hijack the FBFS access token (especially if it were stored in a
well-known location), and send malicious requests.

Since FBFS brings Facebook from the Web browser to the filesystem, there is a
completely different set of features available. Where the website uses a page
covered with links, the filesystem uses a directory with multiple directory
entries. One of the design goals of FBFS is to use filesystem features in a
user-friendly and predictable way. An example of such a feature is the UNIX
permissions system. This attribute can be mapped to Facebook's own permissions
system. As discussed in Section~\ref{subsec:facebookapplications}, the Facebook
permissions system is very flexible. FBFS aims to respect the permissions of
the user and requests minimal permissions at login. For example, if a user
denies the Facebook application access to their photos, the \texttt{photos}
endpoint in the root directory will have neither read nor write access for the
user. If they later allow such permissions, the permissions bits will be
updated accordingly.

The filesystem also utilizes symbolic links to link mutual friends in a
friend's directory. Imagine a user who has two friends, A and B. The user is
friends with both A and B, and A is also friends with B. The user might type
\texttt{ls /friends/A/friends/} and see B's folder in the directory listing.
The user could conceivably recurse through the directory structure infinitely
by following the path \texttt{/friends/A/friends/B/friends/A\dots} forever. By
reporting \texttt{/friends/A/friends/B/} as a symbolic link to
\texttt{/friends/B/}, the problem is solved. If the user wishes to recurse
infinitely, they are allowed to do so, but the filesystem treats the paths as
no more than two levels deep.

These features also work in tandem. FBFS sets the permission bits of the
directories of non-mutual friends-of-friends to zero. Then, when a user lists a
friend's \texttt{friends} directory, it is easy to discern which friends are
mutual, because those friends will be symlinks, while other, non-mutual friends
will be normal directories.

One of the more difficult decisions to make was the naming of files on the
filesystem, in particular for the statuses. Originally I used
ISO-8601~\cite{ISO:1988:IDE} timestamps for the names. However, this proved
undesirable because the names were hard to read and the timestamp information
was already stored in the file metadata. Most other options such as using the
first $n$ characters of the status also did not work because they were not
necessarily unique between multiple statuses. The ultimate decision was to use
the node ID of the status for the filename. Though the node ID is not easy to
read, it both guarantees uniqueness and simplifies metadata lookup, as
described in Section~\ref{sec:implementation}.

The result of such tradeoffs is that the filesystem is usable and fast in the
average use case.

\section{Implementation}
\label{sec:implementation}

FBFS is implemented in two parts: the filesystem, and the Facebook application.
The Facebook application, as discussed in
Section~\ref{subsec:facebookapplications}, is required for interacting with the
Facebook API but requires little developer customization. The filesystem is
where most of the implementation effort was focused. The filesystem is written
in C++, using a variety of libraries.

Since Facebook does not provide an official C++ SDK, I had to resort to
manually building a login flow~\cite{fblogin14} (described in
Section~\ref{subsec:facebookapplications}). When the user mounts FBFS, the
program opens a Web browser for the user to interact with Facebook. Initially,
the program opened the user's default browser (through \texttt{xdg-open}).
Though convenient for the user, this method was impractical because it was
impossible to inspect the current URL and obtain the access token from an
external program. Instead I opted to open a browser using QtWebKit, a class
provided by the Qt Application Framework~\cite{qtproject}. Opening the browser
in this way also allows the program to prevent the user from entering custom
URLs into the browser, which simplifies checking whether the returned access
token is legitimate. If the access token is expired or the user refuses to
install the Facebook application, the program terminates.

Once the application is authenticated, it needs to send a large number of
requests to the Graph API\@. These requests are sent with the
curlcpp library~\cite{curlcpp}, a C++ wrapper around libcurl~\cite{libcurl}.
Responses are parsed by the JSON Spirit library~\cite{jsonspirit}. JSON Spirit
provides a sensible mapping between JSON types and C++ STL types. For example,
JSON objects have an interface similar to \texttt{std::map}, JSON arrays map to
\texttt{std::vector}, and so on.

The biggest performance impact to the filesystem is the number of Graph API
calls that the client must make. This problem is mitigated in two ways. First,
the number of required API calls have been reduced by using node IDs as the
file names. By knowing the node ID, it is possible to look up all of the
metadata about a Graph object in a single API call. Secondly, subsequent API
calls have been sped up by implementing a caching mechanism. Every call to the
Facebook API is stored in a map that associates triples of the form
$(\mathit{node}, \mathit{edge}, \mathit{parameters})$ to the JSON returned by
the API\@. The cache is invalidated periodically, and there is also a method to
skip the cache on specific API calls if fresh data is needed. This drastically
cuts down on the number of API calls that must be made. This is particularly
important because FUSE calls the \texttt{getattr} function quite often, and
it is necessary to fetch information from the API to determine file metadata
such as size or modified time.

The complete code for FBFS is hosted on GitHub~\cite{fbfsgit}.

\section{Evaluation and Future Work}

FBFS has generally reached its initial goals. The filesystem is
usable and searchable with classic UNIX tools.

Performance is the biggest problem currently facing FBFS\@. According to some
testing with Facebook's Graph API Explorer~\cite{graphapiexplorer}, an average
request takes about 250ms, with long requests taking over a second. This
creates an enormous performance impact for even the most basic filesystem
operations. Consider listing a user's status directory with a cold cache. The
initial \texttt{ls} invocation will call FUSE's \texttt{getattr} and
\texttt{readdir}. Then, the filesystem will have to determine the metadata for
each entry in the directory. Assuming that the API returns the first 25
statuses, then a single directory listing can take over 10 seconds! Though
subsequent calls to these status nodes would be cached, a potential
optimization would be to cache based on the node ID instead of the request
itself. This would make lookups far faster, and also more flexible, as a single
edge call such as the one made by \texttt{readdir} could cache up to 25 nodes,
making the subsequent \texttt{getattr} calls nearly instantaneous. I have
included some performance measurements of FBFS with a cold and hot cache in
Table~\ref{tab:perf}. The measurements were made with the UNIX \texttt{time}
command. Though caching is effective, notice the variability in the API
response time, even for the same request.

\begin{table*}[h]
\centering
\caption{FBFS Performance}
\begin{tabular}{lrr} \toprule
Operation & Cold Cache & Warm Cache \\\midrule
\texttt{ls friends}                   &  1.072 & 0.079 \\
                                      &  1.643 & 0.101 \\
                                      &  0.862 & 0.102 \\
                                      &  0.803 & 0.038 \\
                                      &  0.833 & 0.100 \\\midrule
\texttt{ls status}                    & 10.882 & 0.225 \\
                                      & 10.698 & 0.216 \\
                                      &  4.642 & 0.235 \\
                                      & 10.363 & 0.249 \\
                                      & 22.313 & 0.346 \\\midrule
\texttt{cat status/10151702452258380} &  0.307 & 0.146 \\
                                      &  0.137 & 5.849 \\
                                      &  0.150 & 0.140 \\
                                      &  0.362 & 0.350 \\
                                      &  5.645 & 0.127 \\\bottomrule
\end{tabular}
\label{tab:perf}
\end{table*}

Another user-friendly change that could be made in the future is to name the
files with more easily-typable names than their node IDs. However, to keep the
performance benefits from the naming scheme described in
Section~\ref{sec:design}, it would be necessary to keep a map from the
user-displayed name to the node ID for every object in the filesystem, which
would increase the memory overhead.

Due to the sheer amount of information available on Facebook, it is a Herculean
task to account for every type of object and thus create a complete Facebook
filesystem. Though at the moment FBFS only supports viewing friends, statuses,
photos, and albums, with more work it would be possible to support more
endpoints. Each endpoint requires node-specific code in \texttt{getattr} and
\texttt{readdir} in order for the information to be available.

In addition, since Facebook does not offer an official C++ SDK, I hope to
release my Graph API wrapper as a standalone project for use in other C++
applications, as the existing unofficial SDKs are out of date.

\section{Conclusion}

Facebook is an exceedingly popular social network which provides an incredible
amount of information to its users. FBFS allows users to interact with this
information in a new way and enables users to utilize the power of a filesystem
interface to consume and search this information. By bringing the power of the
Graph API to the hands of the user, FBFS brings additional functionality to any
Facebook client.

\bibliographystyle{plain}
\bibliography{final_report}

\end{document}
